% Options for packages loaded elsewhere
\PassOptionsToPackage{unicode}{hyperref}
\PassOptionsToPackage{hyphens}{url}
%
\documentclass[
]{article}
\usepackage{amsmath,amssymb}
\usepackage{lmodern}
\usepackage{iftex}
\ifPDFTeX
  \usepackage[T1]{fontenc}
  \usepackage[utf8]{inputenc}
  \usepackage{textcomp} % provide euro and other symbols
\else % if luatex or xetex
  \usepackage{unicode-math}
  \defaultfontfeatures{Scale=MatchLowercase}
  \defaultfontfeatures[\rmfamily]{Ligatures=TeX,Scale=1}
\fi
% Use upquote if available, for straight quotes in verbatim environments
\IfFileExists{upquote.sty}{\usepackage{upquote}}{}
\IfFileExists{microtype.sty}{% use microtype if available
  \usepackage[]{microtype}
  \UseMicrotypeSet[protrusion]{basicmath} % disable protrusion for tt fonts
}{}
\makeatletter
\@ifundefined{KOMAClassName}{% if non-KOMA class
  \IfFileExists{parskip.sty}{%
    \usepackage{parskip}
  }{% else
    \setlength{\parindent}{0pt}
    \setlength{\parskip}{6pt plus 2pt minus 1pt}}
}{% if KOMA class
  \KOMAoptions{parskip=half}}
\makeatother
\usepackage{xcolor}
\usepackage[margin=1in]{geometry}
\usepackage{color}
\usepackage{fancyvrb}
\newcommand{\VerbBar}{|}
\newcommand{\VERB}{\Verb[commandchars=\\\{\}]}
\DefineVerbatimEnvironment{Highlighting}{Verbatim}{commandchars=\\\{\}}
% Add ',fontsize=\small' for more characters per line
\usepackage{framed}
\definecolor{shadecolor}{RGB}{248,248,248}
\newenvironment{Shaded}{\begin{snugshade}}{\end{snugshade}}
\newcommand{\AlertTok}[1]{\textcolor[rgb]{0.94,0.16,0.16}{#1}}
\newcommand{\AnnotationTok}[1]{\textcolor[rgb]{0.56,0.35,0.01}{\textbf{\textit{#1}}}}
\newcommand{\AttributeTok}[1]{\textcolor[rgb]{0.77,0.63,0.00}{#1}}
\newcommand{\BaseNTok}[1]{\textcolor[rgb]{0.00,0.00,0.81}{#1}}
\newcommand{\BuiltInTok}[1]{#1}
\newcommand{\CharTok}[1]{\textcolor[rgb]{0.31,0.60,0.02}{#1}}
\newcommand{\CommentTok}[1]{\textcolor[rgb]{0.56,0.35,0.01}{\textit{#1}}}
\newcommand{\CommentVarTok}[1]{\textcolor[rgb]{0.56,0.35,0.01}{\textbf{\textit{#1}}}}
\newcommand{\ConstantTok}[1]{\textcolor[rgb]{0.00,0.00,0.00}{#1}}
\newcommand{\ControlFlowTok}[1]{\textcolor[rgb]{0.13,0.29,0.53}{\textbf{#1}}}
\newcommand{\DataTypeTok}[1]{\textcolor[rgb]{0.13,0.29,0.53}{#1}}
\newcommand{\DecValTok}[1]{\textcolor[rgb]{0.00,0.00,0.81}{#1}}
\newcommand{\DocumentationTok}[1]{\textcolor[rgb]{0.56,0.35,0.01}{\textbf{\textit{#1}}}}
\newcommand{\ErrorTok}[1]{\textcolor[rgb]{0.64,0.00,0.00}{\textbf{#1}}}
\newcommand{\ExtensionTok}[1]{#1}
\newcommand{\FloatTok}[1]{\textcolor[rgb]{0.00,0.00,0.81}{#1}}
\newcommand{\FunctionTok}[1]{\textcolor[rgb]{0.00,0.00,0.00}{#1}}
\newcommand{\ImportTok}[1]{#1}
\newcommand{\InformationTok}[1]{\textcolor[rgb]{0.56,0.35,0.01}{\textbf{\textit{#1}}}}
\newcommand{\KeywordTok}[1]{\textcolor[rgb]{0.13,0.29,0.53}{\textbf{#1}}}
\newcommand{\NormalTok}[1]{#1}
\newcommand{\OperatorTok}[1]{\textcolor[rgb]{0.81,0.36,0.00}{\textbf{#1}}}
\newcommand{\OtherTok}[1]{\textcolor[rgb]{0.56,0.35,0.01}{#1}}
\newcommand{\PreprocessorTok}[1]{\textcolor[rgb]{0.56,0.35,0.01}{\textit{#1}}}
\newcommand{\RegionMarkerTok}[1]{#1}
\newcommand{\SpecialCharTok}[1]{\textcolor[rgb]{0.00,0.00,0.00}{#1}}
\newcommand{\SpecialStringTok}[1]{\textcolor[rgb]{0.31,0.60,0.02}{#1}}
\newcommand{\StringTok}[1]{\textcolor[rgb]{0.31,0.60,0.02}{#1}}
\newcommand{\VariableTok}[1]{\textcolor[rgb]{0.00,0.00,0.00}{#1}}
\newcommand{\VerbatimStringTok}[1]{\textcolor[rgb]{0.31,0.60,0.02}{#1}}
\newcommand{\WarningTok}[1]{\textcolor[rgb]{0.56,0.35,0.01}{\textbf{\textit{#1}}}}
\usepackage{longtable,booktabs,array}
\usepackage{calc} % for calculating minipage widths
% Correct order of tables after \paragraph or \subparagraph
\usepackage{etoolbox}
\makeatletter
\patchcmd\longtable{\par}{\if@noskipsec\mbox{}\fi\par}{}{}
\makeatother
% Allow footnotes in longtable head/foot
\IfFileExists{footnotehyper.sty}{\usepackage{footnotehyper}}{\usepackage{footnote}}
\makesavenoteenv{longtable}
\usepackage{graphicx}
\makeatletter
\def\maxwidth{\ifdim\Gin@nat@width>\linewidth\linewidth\else\Gin@nat@width\fi}
\def\maxheight{\ifdim\Gin@nat@height>\textheight\textheight\else\Gin@nat@height\fi}
\makeatother
% Scale images if necessary, so that they will not overflow the page
% margins by default, and it is still possible to overwrite the defaults
% using explicit options in \includegraphics[width, height, ...]{}
\setkeys{Gin}{width=\maxwidth,height=\maxheight,keepaspectratio}
% Set default figure placement to htbp
\makeatletter
\def\fps@figure{htbp}
\makeatother
\setlength{\emergencystretch}{3em} % prevent overfull lines
\providecommand{\tightlist}{%
  \setlength{\itemsep}{0pt}\setlength{\parskip}{0pt}}
\setcounter{secnumdepth}{5}
\usepackage{mathtools}
\usepackage{amsthm}
\usepackage{amssymb}
\usepackage{eufrak}
\usepackage{mathrsfs}
\usepackage{color}
\usepackage[spanish]{babel}
\usepackage{fancyhdr}
\usepackage{array}
\usepackage{subfigure} %para incluir mas de una figura en un solo espacio
\usepackage{graphicx}
\allowdisplaybreaks
\usepackage{float}
\usepackage{booktabs}
\usepackage{longtable}
\usepackage{array}
\usepackage{multirow}
\usepackage{wrapfig}
\usepackage{float}
\usepackage{colortbl}
\usepackage{pdflscape}
\usepackage{tabu}
\usepackage{threeparttable}
\usepackage{threeparttablex}
\usepackage[normalem]{ulem}
\usepackage{makecell}
\usepackage{xcolor}
\newtheorem{teorema}{Teorema}
\newtheorem{lema}[teorema]{Lema}
\newtheorem{corolario}[teorema]{Corolario}
\newtheorem{proposicion}[teorema]{Proposici\'on}
\newtheorem{conjetura}[teorema]{Conjetura}
\newtheorem{definicion}{Definici\'on}
\newtheorem{ejemplo}[teorema]{Ejemplo}
\newtheorem{nota}{Nota}
\ifLuaTeX
  \usepackage{selnolig}  % disable illegal ligatures
\fi
\IfFileExists{bookmark.sty}{\usepackage{bookmark}}{\usepackage{hyperref}}
\IfFileExists{xurl.sty}{\usepackage{xurl}}{} % add URL line breaks if available
\urlstyle{same} % disable monospaced font for URLs
\hypersetup{
  pdfauthor={Andrés Arredondo Cruz (andresabstract@gmail.com); Adriana Haydé Contreras Peruyero (haydeeperuyero@gmail.com); David Alberto García Estrada (david.garcia.e@cinvestav.mx)},
  hidelinks,
  pdfcreator={LaTeX via pandoc}}

\author{Andrés Arredondo Cruz
(\href{mailto:andresabstract@gmail.com}{\nolinkurl{andresabstract@gmail.com}}) \and Adriana
Haydé Contreras Peruyero
(\href{mailto:haydeeperuyero@gmail.com}{\nolinkurl{haydeeperuyero@gmail.com}}) \and David
Alberto García Estrada
(\href{mailto:david.garcia.e@cinvestav.mx}{\nolinkurl{david.garcia.e@cinvestav.mx}})}
\date{}

\begin{document}

\newcommand{\cb}{\color{blue}}
\newcommand{\cg}{\color{green}}
\newcommand{\cvi}{\color{violet}}

\title{ {\sc Escuela Nacional de Estudios Superiores Unidad León}\\
\vspace{1cm}{\sc Centro de Ciencias Matemáticas }\\ 
\vspace{1cm}{\sc UGA-LANGEBIO CINVESTAV}\\
  \vspace{1cm} {Análisis estadístico de datos de Microbioma con R}\\
   \vspace{1.5cm} {Capítulo 8: Análisis univariado de comunidades} \\[2cm]
       \vspace{1.5cm} {Equipo 4}\\
       }

\date{\vspace{5.5cm} Morelia\\
      \vspace{1cm} Septiembre de 2022}

\maketitle

\thispagestyle{fancy}
\newpage

\tableofcontents
\newpage

\hypertarget{introducciuxf3n}{%
\section{Introducción}\label{introducciuxf3n}}

\hypertarget{anuxe1lisis-univariado-de-comunidades}{%
\subsection{Análisis univariado de
comunidades}\label{anuxe1lisis-univariado-de-comunidades}}

Dividimos el estudio de la composición de las comunidades microbianas en
dos grandes componentes: (a)Evaluación de hipótesis sobre diversidad
taxonómica, OTU y Taxones (b) Análisis de diferencias entre grupos. El
primer componente pertenece principalmente a análisis univariado de
comunidades. El segundo puede ser dividido en varias técnicas
multivariadas, como lo son ``clustering'' y ``ordinations'', y la
evaluación de hipótesis de análisis multivariado de disimilitudes.

\#\#8.1 Comparaciones de diversidades entre dos grupos

En nuestro estudio con ratones Vdr-/-, uno de los propósitos es probar
la diferencia de diversidades entre dos grupos (Vdr-/- y ratones de tipo
salvaje) en sitios fecales y cecales. Aquí ilustraremos el análisis de
la comunidad univariante, y compararemos la diversidad de
Shannon(calculado anteriormente en Cap. 6 en las muestras fecales)
usando varias estadísticas de prueba.

\#\#8.1.1 Prueba t de Welch para dos muestras

El estadístico t fue introducido en 1908 por William Sealy Gosset. Una
prueba t de dos muestras se utiliza para probar que las medias de dos
poblaciones son iguales. Se aplica más comúnmente cuando el estadístico
de prueba seguiría una distribución normal. Si los dos grupos tienen la
misma varianza, el estadístico t se puede calcular de la siguiente
manera: \[
t=\frac{\bar X_1 - \bar X_2}{^{Sp} \sqrt{\frac{1}{n_1}+\frac{1}{n_2}}}
\] donde, sp y s22 son el estimador imparcial de la varianza de las
muestras 1 y 2, respectivamente. Cuando las dos muestras tienen
varianzas desiguales y tamaños de muestra desiguales, la prueba t de
Welch se considera más confiable (Ruxton 2006). Por lo tanto, aquí
usamos la prueba t de Welch para nuestros datos de ratón Vdr-/-.
Primero, cargue y transponga el conjunto de datos:

\begin{Shaded}
\begin{Highlighting}[]
\CommentTok{\# IMPORTANTE}
\CommentTok{\# Ajustamos path de trabajo según tu PC}
\NormalTok{path }\OtherTok{\textless{}{-}} \StringTok{"C:/Users/ANDRESARREDONDOCRUZ/"}
\NormalTok{path }\OtherTok{\textless{}{-}} \FunctionTok{paste0}\NormalTok{(path,}\StringTok{"Equipo4/Chapter8/"}\NormalTok{)}
\FunctionTok{setwd}\NormalTok{(path)}
\end{Highlighting}
\end{Shaded}

\begin{Shaded}
\begin{Highlighting}[]
\NormalTok{abund\_table}\OtherTok{=}\FunctionTok{read.csv}\NormalTok{(}\FunctionTok{paste0}\NormalTok{(path,}\StringTok{"data/VdrGenusCounts.csv"}\NormalTok{),}\AttributeTok{row.names=}\DecValTok{1}\NormalTok{,}\AttributeTok{check.names=}\ConstantTok{FALSE}\NormalTok{)}
\NormalTok{abund\_table}\OtherTok{\textless{}{-}}\FunctionTok{t}\NormalTok{(abund\_table)}
\end{Highlighting}
\end{Shaded}

Para incorporar la información del grupo del conjunto de datos
directamente a la comparación, necesitamos administrar los datos. En el
conjunto de datos, la información de id de muestra y grupo está en una
franja de caracteres. Primero los extraemos de allí.

\begin{Shaded}
\begin{Highlighting}[]
\NormalTok{grouping}\OtherTok{\textless{}{-}}\FunctionTok{data.frame}\NormalTok{(}\AttributeTok{row.names=}\FunctionTok{rownames}\NormalTok{(abund\_table),}\FunctionTok{t}\NormalTok{(}\FunctionTok{as.data.frame}\NormalTok{(}\FunctionTok{strsplit}\NormalTok{(}\FunctionTok{rownames}\NormalTok{(abund\_table),}\StringTok{"\_"}\NormalTok{))))}
\NormalTok{grouping}\SpecialCharTok{$}\NormalTok{Location }\OtherTok{\textless{}{-}} \FunctionTok{with}\NormalTok{(grouping, }\FunctionTok{ifelse}\NormalTok{(X3}\SpecialCharTok{\%in\%}\StringTok{"drySt{-}28F"}\NormalTok{, }\StringTok{"Fecal"}\NormalTok{, }\StringTok{"Cecal"}\NormalTok{))}
\NormalTok{grouping}\SpecialCharTok{$}\NormalTok{Group }\OtherTok{\textless{}{-}} \FunctionTok{with}\NormalTok{(grouping,}\FunctionTok{ifelse}\NormalTok{(}\FunctionTok{as.factor}\NormalTok{(X2)}\SpecialCharTok{\%in\%} \FunctionTok{c}\NormalTok{(}\DecValTok{11}\NormalTok{,}\DecValTok{12}\NormalTok{,}\DecValTok{13}\NormalTok{,}\DecValTok{14}\NormalTok{,}\DecValTok{15}\NormalTok{),}\FunctionTok{c}\NormalTok{(}\StringTok{"Vdr{-}/{-}"}\NormalTok{), }\FunctionTok{c}\NormalTok{(}\StringTok{"WT"}\NormalTok{)))}
\NormalTok{grouping }\OtherTok{\textless{}{-}}\NormalTok{ grouping[,}\FunctionTok{c}\NormalTok{(}\DecValTok{4}\NormalTok{,}\DecValTok{5}\NormalTok{)]}
\NormalTok{grouping}
\end{Highlighting}
\end{Shaded}

\begin{longtable}[]{@{}lll@{}}
\toprule()
& Location & Group \\
\midrule()
\endhead
5\_15\_drySt-28F & Fecal & Vdr-/- \\
20\_12\_CeSt-28F & Cecal & Vdr-/- \\
1\_11\_drySt-28F & Fecal & Vdr-/- \\
2\_12\_drySt-28F & Fecal & Vdr-/- \\
3\_13\_drySt-28F & Fecal & Vdr-/- \\
4\_14\_drySt-28F & Fecal & Vdr-/- \\
7\_22\_drySt-28F & Fecal & WT \\
8\_23\_drySt-28F & Fecal & WT \\
9\_24\_drySt-28F & Fecal & WT \\
19\_11\_CeSt-28F & Cecal & Vdr-/- \\
21\_13\_CeSt-28F & Cecal & Vdr-/- \\
22\_14\_CeSt-28F & Cecal & Vdr-/- \\
23\_15\_CeSt-28F & Cecal & Vdr-/- \\
25\_22\_CeSt-28F & Cecal & WT \\
26\_23\_CeSt-28F & Cecal & WT \\
27\_24\_CeSt-28F & Cecal & WT \\
\bottomrule()
\end{longtable}

Repetimos el calculo de la diversidad de Shannon para est tabla, igual
que en el capitulo 6.

\begin{Shaded}
\begin{Highlighting}[]
\CommentTok{\#library(vegan)}
\NormalTok{H}\OtherTok{\textless{}{-}}\FunctionTok{diversity}\NormalTok{(abund\_table, }\StringTok{"shannon"}\NormalTok{) }
\end{Highlighting}
\end{Shaded}

Luego combinamos dataframes de diversidad y agrupación para crear un
nuevo dataframe.

\begin{Shaded}
\begin{Highlighting}[]
\NormalTok{df\_H}\OtherTok{\textless{}{-}}\FunctionTok{data.frame}\NormalTok{(}\AttributeTok{sample=}\FunctionTok{names}\NormalTok{(H),}\AttributeTok{value=}\NormalTok{H,}\AttributeTok{measure=}\FunctionTok{rep}\NormalTok{(}\StringTok{"Shannon"}\NormalTok{,}\FunctionTok{length}\NormalTok{(H)))}
\NormalTok{df\_G }\OtherTok{\textless{}{-}}\FunctionTok{cbind}\NormalTok{(df\_H, grouping)}
\FunctionTok{rownames}\NormalTok{(df\_G)}\OtherTok{\textless{}{-}}\ConstantTok{NULL}
\NormalTok{df\_G}
\end{Highlighting}
\end{Shaded}

\begin{longtable}[]{@{}lrlll@{}}
\toprule()
sample & value & measure & Location & Group \\
\midrule()
\endhead
5\_15\_drySt-28F & 2.460729 & Shannon & Fecal & Vdr-/- \\
20\_12\_CeSt-28F & 2.339725 & Shannon & Cecal & Vdr-/- \\
1\_11\_drySt-28F & 2.228023 & Shannon & Fecal & Vdr-/- \\
2\_12\_drySt-28F & 2.734405 & Shannon & Fecal & Vdr-/- \\
3\_13\_drySt-28F & 2.077282 & Shannon & Fecal & Vdr-/- \\
4\_14\_drySt-28F & 2.466830 & Shannon & Fecal & Vdr-/- \\
7\_22\_drySt-28F & 1.777171 & Shannon & Fecal & WT \\
8\_23\_drySt-28F & 1.999559 & Shannon & Fecal & WT \\
9\_24\_drySt-28F & 1.971996 & Shannon & Fecal & WT \\
19\_11\_CeSt-28F & 1.344813 & Shannon & Cecal & Vdr-/- \\
21\_13\_CeSt-28F & 2.016113 & Shannon & Cecal & Vdr-/- \\
22\_14\_CeSt-28F & 1.955432 & Shannon & Cecal & Vdr-/- \\
23\_15\_CeSt-28F & 1.614456 & Shannon & Cecal & Vdr-/- \\
25\_22\_CeSt-28F & 1.958839 & Shannon & Cecal & WT \\
26\_23\_CeSt-28F & 2.270818 & Shannon & Cecal & WT \\
27\_24\_CeSt-28F & 2.002195 & Shannon & Cecal & WT \\
\bottomrule()
\end{longtable}

A continuación, creamos subconjuntos de datos fecales del nuevo
dataframe.

\begin{Shaded}
\begin{Highlighting}[]
\NormalTok{Fecal\_G}\OtherTok{\textless{}{-}} \FunctionTok{subset}\NormalTok{(df\_G, Location}\SpecialCharTok{==}\StringTok{"Fecal"}\NormalTok{)}
\NormalTok{Fecal\_G}
\end{Highlighting}
\end{Shaded}

\begin{longtable}[]{@{}llrlll@{}}
\toprule()
& sample & value & measure & Location & Group \\
\midrule()
\endhead
1 & 5\_15\_drySt-28F & 2.460729 & Shannon & Fecal & Vdr-/- \\
3 & 1\_11\_drySt-28F & 2.228023 & Shannon & Fecal & Vdr-/- \\
4 & 2\_12\_drySt-28F & 2.734405 & Shannon & Fecal & Vdr-/- \\
5 & 3\_13\_drySt-28F & 2.077282 & Shannon & Fecal & Vdr-/- \\
6 & 4\_14\_drySt-28F & 2.466830 & Shannon & Fecal & Vdr-/- \\
7 & 7\_22\_drySt-28F & 1.777171 & Shannon & Fecal & WT \\
8 & 8\_23\_drySt-28F & 1.999559 & Shannon & Fecal & WT \\
9 & 9\_24\_drySt-28F & 1.971996 & Shannon & Fecal & WT \\
\bottomrule()
\end{longtable}

Ahora los datos están listos para el análisis estadístico. Antes de
realizar la prueba de hipótesis, exploremos la distribución de los
valores de diversidad de Shannon usando la función ggplot()

\begin{Shaded}
\begin{Highlighting}[]
\CommentTok{\#library(ggplot2)}
\CommentTok{\#dividimos el gráfico en dos paneles usando facet\_grid.}
\NormalTok{p}\OtherTok{\textless{}{-}}\FunctionTok{ggplot}\NormalTok{(Fecal\_G, }\FunctionTok{aes}\NormalTok{(}\AttributeTok{x=}\NormalTok{value))}\SpecialCharTok{+}
  \FunctionTok{geom\_histogram}\NormalTok{(}\AttributeTok{color=}\StringTok{"black"}\NormalTok{, }\AttributeTok{fill=}\StringTok{"black"}\NormalTok{)}\SpecialCharTok{+}
  \FunctionTok{facet\_grid}\NormalTok{(Group }\SpecialCharTok{\textasciitilde{}}\NormalTok{ .)}
\end{Highlighting}
\end{Shaded}

El paquete plyr se utiliza para calcular los valores promedio de
diversidad de Shannon de cada grupo

\begin{Shaded}
\begin{Highlighting}[]
\CommentTok{\#library(plyr)}
\CommentTok{\#la función ddply toma input un df y arroja un df de output}
\NormalTok{mu }\OtherTok{\textless{}{-}} \FunctionTok{ddply}\NormalTok{(Fecal\_G, }\StringTok{"Group"}\NormalTok{, summarise, }\AttributeTok{grp.mean=}\FunctionTok{mean}\NormalTok{(value))}
\FunctionTok{head}\NormalTok{(mu)}
\end{Highlighting}
\end{Shaded}

\begin{longtable}[]{@{}lr@{}}
\toprule()
Group & grp.mean \\
\midrule()
\endhead
Vdr-/- & 2.393454 \\
WT & 1.916242 \\
\bottomrule()
\end{longtable}

\begin{Shaded}
\begin{Highlighting}[]
\CommentTok{\#Agregamos las líneas de la media a la gráfica}

\NormalTok{p}\SpecialCharTok{+}\FunctionTok{geom\_vline}\NormalTok{(}\AttributeTok{data=}\NormalTok{mu, }\FunctionTok{aes}\NormalTok{(}\AttributeTok{xintercept=}\NormalTok{grp.mean, }\AttributeTok{color=}\StringTok{"red"}\NormalTok{),}
             \AttributeTok{linetype=}\StringTok{"dashed"}\NormalTok{)}
\end{Highlighting}
\end{Shaded}

\begin{verbatim}
## `stat_bin()` using `bins = 30`. Pick better value with `binwidth`.
\end{verbatim}

\includegraphics{Chapter8_files/figure-latex/unnamed-chunk-9-1.pdf}

En el histograma anterior la eliminación de Vdr de este grupo se
desplaza hacia la derecha en relación con el grupo WT (hacia valores de
diversidad más altos), lo que da como resultado una mayor diversidad.

\begin{Shaded}
\begin{Highlighting}[]
\CommentTok{\#Usamos la prueba t de Welch para probar la hipotesis nula}
\NormalTok{fit\_t }\OtherTok{\textless{}{-}} \FunctionTok{t.test}\NormalTok{(value }\SpecialCharTok{\textasciitilde{}}\NormalTok{ Group, }\AttributeTok{data=}\NormalTok{Fecal\_G)}
\NormalTok{fit\_t}
\end{Highlighting}
\end{Shaded}

\begin{verbatim}
## 
##  Welch Two Sample t-test
## 
## data:  value by Group
## t = 3.5999, df = 5.9206, p-value = 0.01163
## alternative hypothesis: true difference in means between group Vdr-/- and group WT is not equal to 0
## 95 percent confidence interval:
##  0.1517841 0.8026392
## sample estimates:
## mean in group Vdr-/-     mean in group WT 
##             2.393454             1.916242
\end{verbatim}

Para probar la hipótesis nula de que no hay diferencia en la diversidad
de Shannon, se utilizó una prueba t de Welch que dio como resultado un
valor de p = 0,01 (t = 3,6, df = 5,9). Por lo tanto, rechazamos la
hipótesis nula de no diferencia a favor de la alternativa de que las
diversidades de Shannon son diferentes en los dos grupos.

\#\#8.2 Comparaciones de un taxón de interés entre dos grupos

\#8.2.1 Comparación de la abundancia relativa utilizando la prueba de la
suma de rangos de Wilcoxon

Cuando analizamos los datos de abundancia de microbiomas, no es
apropiado sacar inferencias sobre la abundancia total en el ecosistema a
partir de la abundancia de OTUs o abundancia de taxones en las muestras.
Mas bien podemos usar la abundancia relativa en la muestra para inferir
la abundancia relativa de un taxón en el ecosistema. La razón subyacente
es que existe una restricción de composición: todas las abundancias
relativas dentro de una muestra suman a uno, lo que da como resultado
datos de composición residiendo en un simplex (Aitchison 1982, 1986) en
lugar de residir en el espacio euclidiano. Por lo tanto, a menudo es
necesario estandarizar los datos a una escala común para facilitar la
comparación de la abundancia del taxón entre grupos. La forma es dividir
el conteo de taxones por el número total de lecturas en 100 para
convertir la abundancia en el porcentaje de lecturas en la muestra,
escalar los datos a ``el número de taxones por 100 lecturas''.

Cuando seleccionamos un solo taxón específico para probarlo en grupos,
es importante asegúrarse de que el taxón especificado se base en una
hipótesis o teoría para reducir la posibilidad de inflar la tasa de
falsos positivos (es decir, rechaza la hipótesis nula cuando no debería
ser rechazada). Vdr en ratones afecta sustancialmente la diversidad beta
e influencia constantemente taxones bacterianos individuales, como los
Parabacteroides (Wang et al.~2016). En esta sección, ilustramos la
prueba de suma de rangos de Wilcoxon para comparar Bacteroides
bacterianos en el conjunto de datos del ratón Vdr utilizando muestras
fecales.

Primero, verifique la abundancia total en cada muestra.

\begin{Shaded}
\begin{Highlighting}[]
\FunctionTok{apply}\NormalTok{(abund\_table,}\DecValTok{1}\NormalTok{, sum)}
\end{Highlighting}
\end{Shaded}

\begin{verbatim}
## 5_15_drySt-28F 20_12_CeSt-28F 1_11_drySt-28F 2_12_drySt-28F 3_13_drySt-28F 
##           1853           3239           6211           5115           6016 
## 4_14_drySt-28F 7_22_drySt-28F 8_23_drySt-28F 9_24_drySt-28F 19_11_CeSt-28F 
##           2343           2262           7255           5502           5067 
## 21_13_CeSt-28F 22_14_CeSt-28F 23_15_CeSt-28F 25_22_CeSt-28F 26_23_CeSt-28F 
##           2397           3788           9264           2072           6903 
## 27_24_CeSt-28F 
##           6327
\end{verbatim}

Luego, calcule la abundancia relativa dividiendo cada valor por la
abundancia total de la muestra:

\begin{Shaded}
\begin{Highlighting}[]
\CommentTok{\#Calculamos la abundancia relativa}
\NormalTok{relative\_abund\_table }\OtherTok{\textless{}{-}} \FunctionTok{decostand}\NormalTok{(abund\_table, }\AttributeTok{method =} \StringTok{"total"}\NormalTok{)}
\end{Highlighting}
\end{Shaded}

Compruebe la abundancia total en cada muestra para que los cálculos
anteriores sean correctos

\begin{Shaded}
\begin{Highlighting}[]
\FunctionTok{apply}\NormalTok{(relative\_abund\_table, }\DecValTok{1}\NormalTok{, sum)}
\end{Highlighting}
\end{Shaded}

\begin{verbatim}
## 5_15_drySt-28F 20_12_CeSt-28F 1_11_drySt-28F 2_12_drySt-28F 3_13_drySt-28F 
##              1              1              1              1              1 
## 4_14_drySt-28F 7_22_drySt-28F 8_23_drySt-28F 9_24_drySt-28F 19_11_CeSt-28F 
##              1              1              1              1              1 
## 21_13_CeSt-28F 22_14_CeSt-28F 23_15_CeSt-28F 25_22_CeSt-28F 26_23_CeSt-28F 
##              1              1              1              1              1 
## 27_24_CeSt-28F 
##              1
\end{verbatim}

Eche un vistazo a los datos transformados

\begin{Shaded}
\begin{Highlighting}[]
\CommentTok{\#Visualizamos los datos transformados}
\NormalTok{relative\_abund\_table[}\DecValTok{1}\SpecialCharTok{:}\DecValTok{16}\NormalTok{,}\DecValTok{1}\SpecialCharTok{:}\DecValTok{8}\NormalTok{]}
\end{Highlighting}
\end{Shaded}

\begin{verbatim}
##                 Tannerella Lactococcus Lactobacillus Lactobacillus::Lactococcus
## 5_15_drySt-28F 0.256880734  0.17593092    0.05072855               0.0005396654
## 20_12_CeSt-28F 0.020685397  0.22753936    0.18431615               0.0037048472
## 1_11_drySt-28F 0.088391563  0.36982773    0.06987603               0.0040251167
## 2_12_drySt-28F 0.113000978  0.10713587    0.14056696               0.0009775171
## 3_13_drySt-28F 0.165558511  0.39527926    0.05352394               0.0028257979
## 4_14_drySt-28F 0.172428510  0.20102433    0.08749466               0.0004268032
## 7_22_drySt-28F 0.141025641  0.38992042    0.28470380               0.0057471264
## 8_23_drySt-28F 0.072501723  0.27195038    0.32253618               0.0020675396
## 9_24_drySt-28F 0.077062886  0.41948382    0.18175209               0.0025445293
## 19_11_CeSt-28F 0.000000000  0.08328399    0.06512729               0.0013814881
## 21_13_CeSt-28F 0.002503129  0.07217355    0.26658323               0.0000000000
## 22_14_CeSt-28F 0.005279831  0.15311510    0.16710665               0.0007919747
## 23_15_CeSt-28F 0.003993955  0.52536701    0.19635147               0.0026986183
## 25_22_CeSt-28F 0.018339768  0.34121622    0.30164093               0.0043436293
## 26_23_CeSt-28F 0.011734029  0.20338983    0.19716065               0.0014486455
## 27_24_CeSt-28F 0.037142406  0.30235499    0.05768927               0.0020546863
##                Parasutterella Helicobacter   Prevotella Bacteroides
## 5_15_drySt-28F   0.0005396654  0.048030221 0.0652995143 0.147328656
## 20_12_CeSt-28F   0.0000000000  0.000000000 0.0021611609 0.010497067
## 1_11_drySt-28F   0.0001610047  0.000000000 0.0465303494 0.154242473
## 2_12_drySt-28F   0.0007820137  0.002541544 0.0193548387 0.073704790
## 3_13_drySt-28F   0.0003324468  0.003989362 0.0556848404 0.087433511
## 4_14_drySt-28F   0.0000000000  0.013657704 0.0610328638 0.085360649
## 7_22_drySt-28F   0.0000000000  0.001326260 0.0490716180 0.038019452
## 8_23_drySt-28F   0.0016540317  0.000000000 0.0122674018 0.058442453
## 9_24_drySt-28F   0.0001817521  0.000000000 0.0152671756 0.036713922
## 19_11_CeSt-28F   0.0000000000  0.000000000 0.0000000000 0.000000000
## 21_13_CeSt-28F   0.0000000000  0.000000000 0.0004171882 0.002085941
## 22_14_CeSt-28F   0.0000000000  0.000000000 0.0007919747 0.005279831
## 23_15_CeSt-28F   0.0002158895  0.000000000 0.0010794473 0.003346287
## 25_22_CeSt-28F   0.0000000000  0.000000000 0.0033783784 0.009169884
## 26_23_CeSt-28F   0.0002897291  0.000000000 0.0007243228 0.006663769
## 27_24_CeSt-28F   0.0000000000  0.000000000 0.0039513197 0.019282440
\end{verbatim}

Nuestra bacteria de interés \textbf{Bacteroides} está en la columna 8,
subdividamosla:

\begin{Shaded}
\begin{Highlighting}[]
\CommentTok{\#subdividimos a los bacterioides}
\NormalTok{(Bacteroides }\OtherTok{\textless{}{-}}\NormalTok{relative\_abund\_table[,}\DecValTok{8}\NormalTok{]) }
\end{Highlighting}
\end{Shaded}

\begin{verbatim}
## 5_15_drySt-28F 20_12_CeSt-28F 1_11_drySt-28F 2_12_drySt-28F 3_13_drySt-28F 
##    0.147328656    0.010497067    0.154242473    0.073704790    0.087433511 
## 4_14_drySt-28F 7_22_drySt-28F 8_23_drySt-28F 9_24_drySt-28F 19_11_CeSt-28F 
##    0.085360649    0.038019452    0.058442453    0.036713922    0.000000000 
## 21_13_CeSt-28F 22_14_CeSt-28F 23_15_CeSt-28F 25_22_CeSt-28F 26_23_CeSt-28F 
##    0.002085941    0.005279831    0.003346287    0.009169884    0.006663769 
## 27_24_CeSt-28F 
##    0.019282440
\end{verbatim}

Ahora, combine Bacteroides y los dataframes agrupados y crea
subconjuntos de las muestras fecales para un uso posterior.

\begin{Shaded}
\begin{Highlighting}[]
\CommentTok{\#combine Bacterioides y los df agrupados y crea subconjuntos de las muestras}
\NormalTok{Bacteroides\_G }\OtherTok{\textless{}{-}}\FunctionTok{cbind}\NormalTok{(Bacteroides, grouping)}
\FunctionTok{rownames}\NormalTok{(Bacteroides\_G)}\OtherTok{\textless{}{-}}\ConstantTok{NULL}
\NormalTok{Fecal\_Bacteroides\_G }\OtherTok{\textless{}{-}} \FunctionTok{subset}\NormalTok{(Bacteroides\_G, Location}\SpecialCharTok{==}\StringTok{"Fecal"}\NormalTok{)}
\NormalTok{Fecal\_Bacteroides\_G}
\end{Highlighting}
\end{Shaded}

\begin{longtable}[]{@{}lrll@{}}
\toprule()
& Bacteroides & Location & Group \\
\midrule()
\endhead
1 & 0.1473287 & Fecal & Vdr-/- \\
3 & 0.1542425 & Fecal & Vdr-/- \\
4 & 0.0737048 & Fecal & Vdr-/- \\
5 & 0.0874335 & Fecal & Vdr-/- \\
6 & 0.0853606 & Fecal & Vdr-/- \\
7 & 0.0380195 & Fecal & WT \\
8 & 0.0584425 & Fecal & WT \\
9 & 0.0367139 & Fecal & WT \\
\bottomrule()
\end{longtable}

La función boxplot() se usa para generar un diagrama de caja simple de
Bacteroides con el grupo

\begin{Shaded}
\begin{Highlighting}[]
\CommentTok{\#cramos un boxplot con los  grupos }
\CommentTok{\#Diagrama de caja de Bacteroides bacterianos con grupos Vdr{-}/{-} y WT en muestras fecales}
\FunctionTok{boxplot}\NormalTok{(Bacteroides }\SpecialCharTok{\textasciitilde{}}\NormalTok{ Group,}\AttributeTok{data=}\NormalTok{Fecal\_Bacteroides\_G, }\AttributeTok{col=}\FunctionTok{rainbow}\NormalTok{(}\DecValTok{2}\NormalTok{),}\AttributeTok{main=}\StringTok{"Bacteroides in Vdr WT/KO mice"}\NormalTok{)}
\end{Highlighting}
\end{Shaded}

\includegraphics{Chapter8_files/figure-latex/unnamed-chunk-17-1.pdf}

\begin{Shaded}
\begin{Highlighting}[]
\CommentTok{\#con ggplot generamos el boxplot con el siguente codigo}
\FunctionTok{ggplot}\NormalTok{(Fecal\_Bacteroides\_G, }\FunctionTok{aes}\NormalTok{(}\AttributeTok{x=}\NormalTok{Group, }\AttributeTok{y=}\NormalTok{Bacteroides,}\AttributeTok{col=}\FunctionTok{factor}\NormalTok{(Group))) }\SpecialCharTok{+} 
  \FunctionTok{geom\_boxplot}\NormalTok{(}\AttributeTok{notch=}\ConstantTok{FALSE}\NormalTok{)}
\end{Highlighting}
\end{Shaded}

\includegraphics{Chapter8_files/figure-latex/unnamed-chunk-18-1.pdf}

\begin{Shaded}
\begin{Highlighting}[]
\CommentTok{\#Diagrama de caja  bacterias Bacteroides con Vdr−/− y WT en muestras fecales}
\CommentTok{\#generados usando ggplot}
\FunctionTok{ggplot}\NormalTok{(Fecal\_Bacteroides\_G, }\FunctionTok{aes}\NormalTok{(}\AttributeTok{x=}\NormalTok{Group, }\AttributeTok{y=}\NormalTok{Bacteroides)) }\SpecialCharTok{+} 
  \FunctionTok{geom\_boxplot}\NormalTok{(}\AttributeTok{outlier.colour=}\StringTok{"red"}\NormalTok{, }\AttributeTok{outlier.shape=}\DecValTok{8}\NormalTok{, }\AttributeTok{outlier.size=}\DecValTok{4}\NormalTok{) }\CommentTok{\#+ }
\end{Highlighting}
\end{Shaded}

\includegraphics{Chapter8_files/figure-latex/unnamed-chunk-19-1.pdf}

\begin{Shaded}
\begin{Highlighting}[]
  \CommentTok{\#layer(stat\_params = list(binwidth = 2))}
\CommentTok{\#el argumento bidwith=2 genera un error}
\end{Highlighting}
\end{Shaded}

Los diagramas de caja muestran taxones (\textbf{Bacteriodes}) en ratones
de tipo salvaje (WT, n = 3) y para ratones knockout de Vdr (KO, n = 5)

\begin{Shaded}
\begin{Highlighting}[]
\CommentTok{\#Hacemos una prueba de suma de rangos de Wilcoxon}
\NormalTok{fit\_w\_b }\OtherTok{\textless{}{-}} \FunctionTok{wilcox.test}\NormalTok{(Bacteroides }\SpecialCharTok{\textasciitilde{}}\NormalTok{ Group,}\AttributeTok{data=}\NormalTok{Fecal\_Bacteroides\_G)}
\NormalTok{fit\_w\_b}
\end{Highlighting}
\end{Shaded}

\begin{verbatim}
## 
##  Wilcoxon rank sum exact test
## 
## data:  Bacteroides by Group
## W = 15, p-value = 0.03571
## alternative hypothesis: true location shift is not equal to 0
\end{verbatim}

\end{document}
